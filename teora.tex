\chapter{Proto-Teora}
\epigraph{From the beginning of the world this is passed down to us.}{@@@}

\section{Phonology}
\subsection{Sound Changes}

\paragraph{Diphtong reduction (\clong{aj, aw, uj > e, o, y})}???

\paragraph{\clong{a}-Affection (\clong{i, u > e, o / \_ {\textellipsis} a})}
The high vowels \clong{i, u} are lowered by an \clong{a} in the following syllable.

\paragraph{Fricative voicing (\clong{f, s, x > v, z, ɣ > w, ∅, j / V \_})}
The voiceless fricatives \clong{f, s, x} become voiced and \clong{v, ɣ} subsequently vocalize further, whereas
\clong{z} first develops to \clong{h} and is then lost between vowels. This shift is the primary reason for the two major
features of the Teoran subbranch: The vocalization of \clong{f, x} and deletion of \clong{s} produces the many vowels
in contact and the change itself triggers the Teoran chain shift.

\paragraph{Plosive affrication (\clong{p\asp, t\asp, k\asp > f, s, x / V \_})}
Triggered by the loss of fricatives, the aspirates first become affricates \clong{pf, ts, kx} and then develop further to
\clong{f, s, x}. This is inhibited by a preceding nasal, so the combinations \clong{mp\asp, nt\asp, ŋk\asp} remain
unchanged. The voiced plosives do not participate in this chain shift.

\paragraph{Diphthong dissimilation (\clong{aw, aj > o, e / \_ {\textellipsis} aw, aj})}

\paragraph{Initial \clong{x} weakens (\clong{x > h / \# \_})}

\subsection{Phonemic inventory}

\begin{table}[h]
	\centering
	\captionabove{Consonant Inventory of Proto-Teora}
	\begin{tabular}{l c c c c c c}
		\toprule
		                                    &          \multicolumn{3}{c}{Initial}          &             \multicolumn{3}{c}{Medial and Final}             \\
		\cmidrule(r){2-4} \cmidrule(r){5-7} & Labial        & Dental        & Velar         & Labial             & Dental             & Velar              \\ \midrule
		Nasals                              & \clong{m}     & \clong{n}     & \clong{ŋ}     & \clong{m}          & \clong{n}          & \clong{ŋ}          \\
		Plosives                            & \clong{p\asp} & \clong{t\asp} & \clong{k\asp} & \clong{f (mp\asp)} & \clong{s (nt\asp)} & \clong{x (ŋk\asp)} \\
		                                    & \clong{b}     & \clong{d}     & \clong{g}     & \clong{b (mb)}     & \clong{d (nd)}     & \clong{g (ŋg)}     \\
		Fricatives                          & \clong{f}     & \clong{s}     & \clong{x}     & \clong{w}          & \clong{∅}          & \clong{j}          \\
		Lateral                             &               & \clong{l}     &               &                    & \clong{l}          &                    \\
		Trill                               &               & \clong{r}     &               &                    & \clong{r}          &                    \\ \bottomrule
	\end{tabular}
\end{table}
\subsection{Morphonology}