\chapter{Proto-Tisauric}
\epigraph{Thus my hypothesis is that the tongues of the commoners are not wholly unrelated to old tongue of knowledge}{@@@}

\section{Phonology}
\subsection{Phonemes}
Proto-Tisauric has a relatively small and symmetrical consonant inventory distinguishing a labial, dental and velar series.
Each series has two different kinds of plosives, reconstructed as voiceless aspirate and voiced due to the affrication of the
first kind in Teora Falachin and the rhotacism triggered by the second one in Classical Tiskaur, but it is unclear whether both
aspiration and voice were used to distinguish between the plosives or whether one of the features developed secondarily.
The apart from assimilations the nasals remain comparatively constant in both branches, as do the liquids. The fricatives are
easiest reconstructed from Classical Tiskaur, whereas in the southern branch they are lost or vocalized in various ways,
resulting in the characteristic diphthongs and vowels in hiatus in this branch.

\begin{table}[h]
	\centering
	\captionabove{Consonant Inventory of Proto-Tisauric}
	\begin{tabular}{l c c c}
		\toprule
		           & Labial        & Dental        & Velar         \\ \midrule
		Nasals     & \clong{m}     & \clong{n}     & \clong{ŋ}     \\
		Plosives   & \clong{p\asp} & \clong{t\asp} & \clong{k\asp} \\
		           & \clong{b}     & \clong{d}     & \clong{g}     \\
		Fricatives & \clong{f}     & \clong{s}     & \clong{x}     \\
		Lateral    &               & \clong{l}     &               \\
		Trill      &               & \clong{r}     &               \\ \bottomrule
	\end{tabular}
\end{table}

It is commonly thought that even though both branches of Tisauric show at least \clong{i, e, a, o, u}.
Proto-Tisauric started out with only three vowels \clong{i, a, u}. The situation is somewhat obscured 
because all decedents eventually begin to accept new words with \clong{e, o} in positions without historical 
justification, but since the branches developed them in different circumstances no proto-phonemes 
\clong{\proto e, o} can be reconstructed.

\begin{table}[h]
	\centering
	\captionabove{Vowel Inventory of Proto-Tisauric}
	\begin{tabular}{l c c c}
		\toprule
		      & Front          & Mid       & Back           \\ \midrule
		Close & \clong{i, (j)} &           & \clong{u, (w)} \\
		Open  &                & \clong{a} &                \\ \bottomrule
	\end{tabular}
\end{table}

\subsection{Syllable Shape}
The basic syllable shape is \clong{\normalfont\textsc{(c)v(c)}}, though only stops are not permitted to serve as a coda. If only one consonant stands between two vowels, it is universally analyzed as belonging to the second syllable.

The vowels can form three falling diphthongs \clong{ai, au} and \clong{ui}.
Furthermore two vowels that could form a diphthong cannot occur in hiatus but form their diphthong
even across morpheme boundaries. The remaining combinations \clong{iu, ia} and \clong{ua} can stand in hiatus, but this is not
permitted for two consecutive vowels of the same quality \clong{**ii, uu, aa}.

\subsection{Allophony}
Due ot its nature as a unattested proto-language, no allophonic variation proper is known for this language.
It is noteworthy however that the lowering effect that \clong{x} had on preceding vowels
in the northern daughter languages seem to indicate that it was indeed realized as something
close to \clong{[x]} rather than \clong{[ç]}, even after front vowels.
